\setlength{\headheight}{15pt}
\documentclass[12pt]{mcmthesis}
\mcmsetup{CTeX = false, 
        tcn = 2211494, problem = C,   % 修改控制号和选题号
        sheet = true, titleinsheet = true, keywordsinsheet = true,
        titlepage = false, abstract = true}
\usepackage{palatino}
\usepackage{lipsum}
\usepackage{amsmath}  % 此处引入数学公式的包
\usepackage{graphicx} % 用于插入图片
 
% 控制页 %%%%%%%%%%%%%%%%%%%%%%%%%%%%%%%%%%%%%%%%%%%%%%%%%%%%%%%%%%%%%%%
% 论文标题
\title{The \LaTeX{} Template for MCM Version \MCMversion}  % 修改标题
\date{\today}
 
\begin{document}
\begin{abstract}
  % 摘要部分
  Here is the main abstract part. \\ Go to the next line.
  % 关键词
  \begin{keywords}
    keyword1; keyword2
  \end{keywords}
\end{abstract}

% 目录页 %%%%%%%%%%%%%%%%%%%%%%%%%%%%%%%%%%%%%%%%%%%%%%%%%%%%%%%%%%%%%%%%
\maketitle         % 控制序列
\tableofcontents   % 生成目录
\newpage

% 基础用法 %%%%%%%%%%%%%%%%%%%%%%%%%%%%%%%%%%%%%%%%%%%%%%%%%%%%%%%%%%%%%%%

% 标题 -----------------------------------------
\section{Introduce}
\subsection{Background}

Market traders buy and sell volatile assets to maximize their total return. Quantitative trading is a great means to achieve this.

The so-called quantitative trading means that investors use computer technology, financial engineering modeling, and other means to make investment decisions and execute trading strategies in strict accordance with the rules set to determine the amount and price of volatile assets to be bought and sold.

With the development of computer technology and modern financial theory, quantitative trading, which realizes automatic trading of securities with the help of electronic technology, has come into being. Quantitative trading has many advantages:
\begin{itemize}
  \item Multiply efficiency by using historical data for strategy checking,
  \item Capture trading opportunities in real-time across the market, dramatically improving profitability,
  \item Allows for more objective measurement of trading results,
  \item Access to profit opportunities that are difficult to find by human hands alone.
\end{itemize}
With these advantages, quantitative trading has received widespread attention from the industry since its inception in the 1970s and has grown at an alarming rate.
\subsection{Problem Restatement}
% 无序符号----------------------------------------
\begin{itemize}
  \item minimizes the discomfort to the hands, or
  \item maximizes the outgoing velocity of the ball.
\end{itemize}
We focus exclusively on the second definition.
\begin{itemize}
  \item the initial velocity and rotation of the ball,
  \item the initial velocity and rotation of the bat,
  \item the relative position and orientation of the bat and ball, and
  \item the force over time that the hitter hands applies on the handle.
\end{itemize}


% 图片 -------------------------------------------------------

% 插入图片,不推荐, 图片路径可使用相对路径和绝对路径, 无排序
\includegraphics[scale=0.2]{404.png}  % scale 缩放比例 0.1

%使用figure浮动窗体

\begin{figure}[ht]   % ht 表示here  top
  \centering
  \includegraphics[scale=0.2]{404.png}
  \caption{this is a figure demo}
  \label{fig:label}
\end{figure}





\subsection{Overview}
Here is a example to cite the referenced article\cite{konishi:1999ab}. \\
Another article\cite{refName}.
\section{Assumptions and Justifications}
\subsection{Model Preparation}
% 行内公式
This is an inline formula. $a = \sqrt{b + c}$.
% 行间公式, 带编号
\begin{equation}
  E = mc^2
\end{equation}
\begin{equation}
  F = ma
\end{equation}
% 定理
\begin{Theorem} \label{thm:latex}
  \LaTeX
\end{Theorem}
% 引理
\begin{Lemma} \label{thm:tex}
  \TeX .
\end{Lemma}
% 证明
\begin{proof}
  The proof of theorem.
\end{proof}

% 图片 -------------------------------------------------------

% 插入图片,不推荐, 图片路径可使用相对路径和绝对路径, 无排序
\includegraphics[scale=0.2]{404.png}  % scale 缩放比例 0.1

%使用figure浮动窗体

\begin{figure}[ht]   % ht 表示here  top
  \centering
  \includegraphics[scale=0.2]{404.png}
  \caption{this is a figure demo}
  \label{fig:label}
\end{figure}



% 表格 -------------------------------------------------------


\subsubsection{Table-1}
\begin{tabular}{|l|c|r|}
  \hline
  OS         & Release  & Editor    \\
  \hline
  Windows    & MikTeX   & TexMakerX \\
  \hline
  Unix/Linux & teTeX    & Kile      \\
  \hline
  Mac OS     & MacTeX   & TeXShop   \\
  \hline
  General    & TeX Live & TeXworks  \\
  \hline
\end{tabular}

\subsubsection{Table-2}
\begin{tabular}{|r r|}
  % r代表row, 使用 | 来划分,如果 r | r中间的|去掉,那么列之间元素无直线划分
  \hline
  1234 & 5678 \\ \hline
  1    & 2    \\ \hline
  3    & 4    \\ \hline
\end{tabular}
\subsubsection{Table-3}
\begin{tabular}{ll}
  \hline
  symbols & definitions                       \\
  \hline
  $v_i$   & velocity of ball before collision \\
  $v_f$   & velocity of ball after collision  \\
  $V_f$   & velocity of bat after collision   \\
  $S$     & the shear modulus the bat         \\
  $Y$     & Young’s modulus of the bat        \\
  \hline
\end{tabular}
% 文献引用 %%%%%%%%%%%%%%%%%%%%%%%%%%%%%%%%%%%%%%%%%%%%%%%%%%%
\subsection{Cite}
Here is a example to cite the referenced article\cite{konishi:1999ab}. \\
Another article\cite{refName}.
\[
  \begin{pmatrix}{*{20}c}
    {a_{11} } & {a_{12} } & {a_{13} } \\
    {a_{21} } & {a_{22} } & {a_{23} } \\
    {a_{31} } & {a_{32} } & {a_{33} } \\
  \end{pmatrix}
  = \frac{{Opposite}}{{Hypotenuse}}\cos ^{ - 1} \theta \arcsin \theta
\]
\lipsum[9]
\[
  p_{j}=\begin{cases} 0,              & \text{if $j$ is odd}  \\
              r!\,(-1)^{j/2}, & \text{if $j$ is even}
  \end{cases}
\]
\lipsum[10]
\[
  \arcsin \theta  =
  \mathop{{\int\!\!\!\!\!\int\!\!\!\!\!\int}\mkern-31.2mu
    \bigodot}\limits_\varphi
  {\mathop {\lim }\limits_{x \to \infty } \frac{{n!}}{{r!\left( {n - r}
          \right)!}}} \eqno (1)
\]
\section{Solution to Problem1 }
\subsection{...}
\lipsum[11]
\section{Solution to Problem2}
\subsection{...}
\lipsum[6]
\section{Solution to Problem3}
\subsection{...}
\lipsum[9]
\section{Solution to Problem4}
\subsection{...}
\lipsum[6]
\section{Sensitivity Analysis}
\lipsum[6]
\section{Strengths and weaknesses}
\lipsum[12]
\subsection{Strengths}
\begin{itemize}
  \item \textbf{Applies widely}\\
        This  system can be used for many types of airplanes, and it also
        solves the interference during  the procedure of the boarding
        airplane,as described above we can get to the  optimization
        boarding time.We also know that all the service is automate.
\end{itemize}
\subsection{Weakness}
\begin{itemize}
  \item \textbf{Improve the quality of the airport service}\\
        Balancing the cost of the cost and the benefit, it will bring in
        more convenient  for airport and passengers.It also saves many
        human resources for the airline.
\end{itemize}
\section{Reference}
% 引用文献 %%%%%%%%%%%%%%%%%%%%%%%%%%%%%%%%%%%%%%%%%%%%%%%%%%%%%%%
\newpage
\bibliography{article}      % 指定article 代表同目录下的article.bib文件 
\bibliographystyle{ieeetr}  % 定义文献引用的格式
% 附录 %%%%%%%%%%%%%%%%%%%%%%%%%%%%%%%%%%%%%%%%%%%%%%%%%%%%%%%%%
\begin{appendices}
  \section{First appendix}
  \lipsum[13]
  Here are simulation programmes we used in our model as follow.\\
\end{appendices}
\end{document}